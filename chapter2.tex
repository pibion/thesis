%
% Modified by Sameer Vijay
% Last Change: Wed Jul 27 2005 13:00 CEST
%
%%%%%%%%%%%%%%%%%%%%%%%%%%%%%%%%%%%%%%%%%%%%%%%%%%%%%%%%%%%%%%%%%%%%%%%%
%
% Sample Notre Dame Thesis/Dissertation
% Using Donald Peterson's ndthesis classfile
%
% Written by Jeff Squyres and Don Peterson
%
% Provided by the Information Technology Committee of
%   the Graduate Student Union
%   http://www.gsu.nd.edu/
%
% Nothing in this document is serious except the format.  :-)
%
% If you have any suggestions, comments, questions, please send e-mail
% to: ndthesis@gsu.nd.edu
%
%%%%%%%%%%%%%%%%%%%%%%%%%%%%%%%%%%%%%%%%%%%%%%%%%%%%%%%%%%%%%%%%%%%%%%%%

%
% Chapter 2
%

\chapter{TRANSFER REACTIONS AND NUCLEAR MATRIX ELEMENTS}
\label{chap:nucl}

\section{Nucleon States and the Impact they Have on NME}
\subsection{NME calculations and valence States}
show how sensitive NME calculations are to occupied valence states

\subsection{Single Nucleon Transfer and Pickup Reactions}
Discuss single nucleon and pickup reactions and what they show about valence state occupation

\subsection{Limitations}
Discuss the limitations faced when all you know about is orbital occupancy

\section{Nucleon-nucleon correlations and the impact they have on NME}
discuss correlation effects on NME
momentum?  momentum of correlated states relevant in \zvbb? compared to momentum of correlated states we probe with our transfer reaction?

\subsection{Two-nucleon transfer reactions}
Discuss current work on two-neutron transfer reactions
INTERPRET

\subsection{Modeling two-proton Transfers}
Discuss theory of two-proton transfer reactions
\begin{equation}
M\sim\langle\psi_n|V_T|\psi_{^3He}\rangle
\end{equation}
Where $V_T$ is the transfer operator and $\psi$ are elastic scattering wave functions.
Assume that nuclear elastic scattering is the largest contribution to the nuclear reaction and use 1st order perturbation theory to generate $V_T$ from the bound-state wave functions.
To get $\psi$ (?)
\begin{enumerate}
\item treat two particles individually
\item treat two protons as bound cluster
\end{enumerate}
treating particles individually is too complicated
when using cluster model, can adjust $V_0$ of well to match ?? binding energy and adjust $r_0$ to adjust range and $a_0$ to adjust ??.

cluster model's prediction of 0 degree cross section is sensitive to these parameters, but the relative cross sections of f-p-g shell nuclei do not depend strongly on the parameters.

Discuss experimental difficulties of two-proton transfer reactions and introduce NSL as a good place to do them

% % uncomment the following lines,
% if using chapter-wise bibliography
%
% \bibliographystyle{ndnatbib}
% \bibliography{example}
