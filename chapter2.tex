%
% Modified by Sameer Vijay
% Last Change: Wed Jul 27 2005 13:00 CEST
%
%%%%%%%%%%%%%%%%%%%%%%%%%%%%%%%%%%%%%%%%%%%%%%%%%%%%%%%%%%%%%%%%%%%%%%%%
%
% Sample Notre Dame Thesis/Dissertation
% Using Donald Peterson's ndthesis classfile
%
% Written by Jeff Squyres and Don Peterson
%
% Provided by the Information Technology Committee of
%   the Graduate Student Union
%   http://www.gsu.nd.edu/
%
% Nothing in this document is serious except the format.  :-)
%
% If you have any suggestions, comments, questions, please send e-mail
% to: ndthesis@gsu.nd.edu
%
%%%%%%%%%%%%%%%%%%%%%%%%%%%%%%%%%%%%%%%%%%%%%%%%%%%%%%%%%%%%%%%%%%%%%%%%

%
% Chapter 2
%

\chapter{TRANSFER REACTIONS AND NUCLEAR MATRIX ELEMENTS}
\label{chap:nucl}

\section{Shell Model of the Nucleus}
Using H.O. eigenstates to describe nucleons.
\begin{itemize}
\item nuclear potential well - is this something we can understand independently?  Certainly the radius is.
\item solutions to finite square well in terms of H.O. eigenstates and a diagram of energy levels
\item the energy levels give correct shell closures when spin-orbit coupling is introduced
\item so H.O. eigenstates are a reasonable way to describe nucleons
\end{itemize}

Looking at the valence nucleons to understand the whole nucleus.
\begin{itemize}
\item and because the nucleons couple so strongly into \zp pairs, describing the unpaired nucleons often accurately describes the entire nucleus
\item give a simple example?  \He{3} might be useful?
\item show level filling for \Ge{74} and point out the valence nucleons are f, p, g
\end{itemize}
 

\section{Nucleon States and the Impact they Have on NME}
\subsection{NME calculations and valence States}
\begin{itemize}
\item (assume that) NME calculations are sensitive to occupied valence states
\item can investigate valence state occupation with transfer reactions
\item \Ge{76} occupancies were experimentally determined and adjusting the energy levels to match changed the NME by a factor of 2 \cite{0vbbReview}
\end{itemize}


\section{Nucleon-nucleon correlations and the impact they have on NME}
\begin{itemize}
\item \zvbb occurs on correlated neutron pairs
\item certain pairing in the nucleus can inhibit \zvbb
\item investigate pair correlation with two-nucleon transfer reactions
\item some systems don't have all their \zp in the ground state - Xe
\item discuss neutron transfer experiments done on \GeTargets
\item wish to investigate proton transfer on \GeTargets
\end{itemize}


\section{Modeling two-proton Transfers}
Discuss DWBA theory of two-proton transfer reactions
\begin{equation}
M\sim\langle\psi_n|V_T|\psi_{^3He}\rangle
\end{equation}
Where $V_T$ is the transfer operator and $\psi$ are elastic scattering wave functions.
Assume that nuclear elastic scattering is the largest contribution to the nuclear reaction and use 1st order perturbation theory to generate $V_T$ from the bound-state wave functions.
To get $\psi$ (?)
\begin{enumerate}
\item treat two particles individually
\item treat two protons as bound cluster
\end{enumerate}
treating particles individually is too complicated
when using cluster model, can adjust $V_0$ of well to match ?? binding energy and adjust $r_0$ to adjust range and $a_0$ to adjust ??.

cluster model's prediction of 0 degree cross section is sensitive to these parameters, but the relative cross sections of f-p-g shell nuclei do not depend strongly on the parameters.

Discuss experimental difficulties of two-proton transfer reactions and introduce NSL as a good place to do them

% % uncomment the following lines,
% if using chapter-wise bibliography
%
% \bibliographystyle{ndnatbib}
% \bibliography{example}
