%
% Modified by Sameer Vijay
% Last Change: Tue Jul 26 2005 13:00 CEST
%
%%%%%%%%%%%%%%%%%%%%%%%%%%%%%%%%%%%%%%%%%%%%%%%%%%%%%%%%%%%%%%%%%%%%%%%%
%
% Sample Notre Dame Thesis/Dissertation
% Using Donald Peterson's ndthesis classfile
%
% Written by Jeff Squyres and Don Peterson
%
% Provided by the Information Technology Committee of
%   the Graduate Student Union
%   http://www.gsu.nd.edu/
%
% Nothing in this document is serious except the format.  :-)
%
% If you have any suggestions, comments, questions, please send e-mail
% to: ndthesis@gsu.nd.edu
%
%%%%%%%%%%%%%%%%%%%%%%%%%%%%%%%%%%%%%%%%%%%%%%%%%%%%%%%%%%%%%%%%%%%%%%%%


%
% Chapter 1
%

\chapter{NEUTRINO PHYSICS AND ITS DEPENDENCE ON NUCLEAR PHYSICS}
\label{chap:0vbb}
\section{A Brief Overview of Neutrino Physics}
\begin{comment}
Neutrino physics is fascinating because ??????.
Maybe I will talk about particles in general - but no, that's probably a bit too much.
I will mention the standard model table and talk about why neutrinos are important in physics today?

The neutrino was first proposed as a undetectable particle that carried away energy in nuclear decay processes \cite{Pauli}.  Twenty-six years later (fact check!), Reines and Cowan detected inverse beta decay to detect antineutrinos streaming out of a nearby nuclear reactor \cite{poltergeist}.  Detecting neutrinos is difficult: with an interaction volume of 10$^{26}$ potential targets, Reines and Cowan saw 1 event per week (FACT CHECK!!) \cite{poltergeist}.  Simply confirming the existence of the particle hypothesized to participate in beta decay was significant enough to merit the Nobel Prize \cite{CowanNobel}.

Figure: beta decay spectrum, with and without neutrino.  See F.A. Scott, Phys Rev 48, 391 (1935)

While detecting neutrinos is a difficult endeavor, it's also a lucrative one: particles that interact extremely rarely carry information about where they're created, no matter what they have to travel through to get to us.  With a detector that counts neutrinos, one can begin to imagine interrogating the cosmos: ``How many neutrinos are you making?  And you?  And you?''  Bahcahll was interested in finding out how many neutrinos the sun made.  And so he and Ray Davis set out to make a good neutrino counter. 

(AMY!  You're neglecting to talk about the experiment that demonstrated the left-handedness of neutrinos by Goldhaber in 1958.  This is kind of relevant to \zvbb!)
\end{comment}

The neutrino, a massless, chargeless particle, was first proposed by Pauli \cite{Pauli} as a means to preserve momentum conservation in nuclear beta decay.  The hypothesized process was
\begin{equation}
n \rightarrow p + e^- + \overline{v_e},
\end{equation}
where a neutron $n$ decays into a proton $p$, an electron $e^-$, and also an electron anti-neutrino $\overline{v_e}$, allowing the continuous electron energy spectrum that was observed. The existence of this difficult-to-detect particle was not confirmed until 26 years later, when Reines and Cowan used a nuclear reactor as a source of antineutrinos and observed inverse beta decay of proton targets \cite{poltergeist}.  Since the first detection of electron anti-neutrinos, an impressive amount has been learned about these elusive particles.  This chapter will begin by discussing what is currently known about neutrinos, in particular that there are three, unique flavors and that while they are very light, they do have mass.  Understanding how neutrinos get their mass in the Standard Model framework and the experiments currently underway to help determine the nature of the neutrino occupies the rest of the chapter.

\subsection{Neutrino Oscillation - the beginning}
Neutrinos interact very weakly with matter, and while this makes their detection difficult, it also makes them a potentially valuable source of information.  Studying the interior of systems that produce neutrinos becomes possible with a neutrino detector, while other forms of radiation would be absorbed long before reaching scientist's detectors.  Ray Davis and John Bahcahll recognized the neutrino could be used to test the theory that nuclear fusion was the sun's energy source.  The neutrino detector built by Ray Davis consisted of 100,000 [CITE] gallons of $^{37}$Cl, which was readily available as dry cleaning fluid.  Solar neutrinos interacting with $^{37}$Cl that inverse-beta decay into $^{37}$Ar leave a detectable signal, as $^{37}$Ar is radioactive.  Years of careful data taking yeilded a count of $\sim$7 neutrinos per two weeks \cite{Davis}, only $\frac{1}{3}$ the rate predicted by Bahcall [CITE].  Further refinements to the experiment and to the calculations confirmed the discrepancy [CITE].

While the Davis experiment continued to collect data, other experiments began that explored the neutrino itself.  Originally imagined as a single particle, it was found that there are three distinct flavors of neutrinos, each associated with a lepton partner.  It is important to note that the decay 
\begin{equation}
p \rightarrow n + \overline{l} + v_l,
\end{equation}
where $l$ is an electron, muon, or tau, requires the proton to have enough momentum relative to the target neutron to create the lepton $l$.  Even the highest-energy nuclear reactions in the sun provide only $\sim$11~MeV, so that the sun can produce only electron neutrinos.  Energetic pion beams at Brookhaven National Laboratory (BNL) provided the first direct measurement of muon neutrinos \cite{muonNeutrino}.  Later, accelerated proton beams at Fermilab were used to successfully detect tau neutrinos \cite{tauNeutrino}.  The inclusion of the neutrino into the Standard Model as a participant in weak interactions mediated by the $W^{\pm}$ and $Z^0$ bosons suggested that experiments determining the lifetime of the $Z^0$ boson could determine the number of interacting neutrinos.  An electron-positron collision experiment at CERN measured the number of neutrino flavors to be 3 to a certainty of ?? [CITE].

That there are three flavors of neutrinos, each associated with a different-mass lepton, is significant because the Davis experiment was sensitive only to electron neutrinos.  Other radiochecmical neutrino experiments, also only sensitive to electron neutrinos, confirmed Davis' results [CITE].  An idea suggested by Pontecorvo long before neutrino detection, that neutrinos have mass, showed a way forward.  Neutrinos had been incorporated into the Standard Model as massless, making it impossible for their flavor to vary with time.  If neutrinos were massive, neutrinos could change flavor.  The hypothesis was that the radiochemical experiments, sensitive only to electron neutrinos, were measuring a deficit because $\frac{2}{3}$ of the electron neutrinos from the sun had changed flavor and could not be detected.  SNO, an experiment designed to be sensitive to all three neutrino flavors, measured the predicted number of solar neutrinos [CITE], confirming that neutrino flavors change with time and therefore that they must be massive.

That neutrino flavor oscillation implies a massive neutrino can be illustrated by imagining mixing between only two neutrino flavors.  Then the 



\subsection{Neutrino Oscillation - current known mixing parameters}
\begin{comment}
Talk about current constraints on parameters
Talk about experiments that constrain these parameters?  This seems far afield, maybe.
\end{comment}

\begin{align}
|\psi_e\rangle &= U_{e1}|\psi_1\rangle + U_{e2}|\psi_2\rangle \\
|\psi_{\mu}\rangle &= U_{{\mu}1}|\psi_1\rangle + U_{{\mu}2}|\psi_2\rangle 
\end{align}
where $\hat{H}|\psi_1\rangle = E_1|\psi_1\rangle$ and $\hat{H}|\psi_2\rangle = E_1|\psi_2\rangle$.  Then for a neutrino with an initial state $|\psi_e\rangle$, the probability of detecting a muon neutrino is
\begin{align}
P(\nu_e\rightarrow\nu_{\mu}) &=  |\langle\psi_{\mu}|\hat{T}|\psi_e\rangle|^2 \\
                             &=  |\langle\psi_{\mu}|e^{i\hat{H}t / \hbar}|\psi_e\rangle|^2 \\
                             &=  U_{e1}U_{e1}U_{{\mu}1}U_{{\mu}2} \times \frac{\cos(E_1 - E_2)t/\hbar)}{2} 
\end{align}
While this calculation is not accurate in many ways, it illustrates several important points about neutrino oscillation experiments.  The first is that the oscillation of detection probability is sensitive to the difference of the masses squared.  Neutrino oscillation experiments are therefore sensitive to the differences between neutrino masses but not to the absolute mass scale.  Several limits on the absolute mass scale of the neutrino exist, notably from cosmology [CITE] and from efforts to measure the mass of the electron neutrino by very carefully measuring the endpoint of beta decay [CITE].  These limits constrain the total mass to be less than 2~eV.  

The neutrino mixing matrix $U$ can be written with three angles,
\begin{equation}
\begin{bmatrix}
c_{12}c_{13} & s_{12}c_{13} & s_{13}e^{-i\delta} \\
-s_{12}c_{23}-c_{12}s_{23}s_{13}e^{i\delta} & c_{12}c_{23}-s_{12}s_{23}s_{13}e^{i\delta} & s_{23}c_{13} \\
s_{12}s_{23}-c_{12}c_{23}s_{13}e^{i\delta} & -c_{12}s_{23}-s_{12}c_{23}s_{13}e^{i\delta} & c_{23}c_{13} 
\end{bmatrix}
\times
\begin{bmatrix}
1 & 0 & 0 \\
0 & e^{\alpha_{21} / 2} & 0 \\
0 & 0 & e^{\alpha_{31} / 2},
\end{bmatrix}
\end{equation}

where $c_{ij} = \cos{\theta_{ij}}$ and $s_{ij} = \sin{\theta_{ij}}$, $\delta$ is the Dirac CP-violating phase, and the Majorana CP-violating phases $\alpha_{ij}$ are only relevant if the neutrino is a Majorana particle as discussed in {\sect}~\ref{sec:majorana}.  Several generations of long-baseline neutrino experiments using solar, atmospheric, and reactor neutrinos have impressively constrained the mixing parameters and mass differences.  A summary of the parameters is given in {\tab}~\ref{tab:neutrinoParameters}.
\begin{table*}
\centering
\begin{tabular}{lll}\toprule
Parameter & Best Fit ($\pm$ 1$\sigma$) & 3$\sigma$ \\
\midrule
${\Delta}m^2_{\odot}$ [$10^{-5}$ eV$^2$] & $7.58^{+0.22}_{-0.26}$ & 6.99 - 8.18 \\
$|{\Delta}m^2_A|$ [$10^{-3}$ eV$^2$] & $2.35^{+0.12}_{-0.09}$ & 2.06 - 2.67 \\
$\sin^2{\theta_{12}}$ & 0.306 $(0.312)^{+0.018}_{-0.015}$ & 0.259 (0.265) - 0.359 (0.364) \\  
$\sin^2{\theta_{23}}$ & $0.42^{+0.08}_{-0.03}$ & 0.34 - 0.64 \\  
$\sin^2{\theta_{13}}$ & 0.021 $(0.025)^{+0.007}_{-0.008}$ & 0.001 (0.005) - 0.044 (0.050) \\   
$\sin^2{\theta_{13}}$ & 0.0251 $\pm$ 0.0034 & 0.015 - 0.036 \\
\bottomrule  
\end{tabular}
\caption{From PDG.}
\label{tab:neutrinoParameters}
\end{table*}
Long-baseline neutrino experiments have provided a comprehensive picture of neutrino mixing, but they cannot provide access to important information about the neutrino such as the CP-violating phases or the nature of its mass.  These will be discussed in the next section.


\section{Majorana vs Dirac}
\label{sec:mVd}
\begin{comment}
Discuss mechanisms by which neutrinos could get their mass.  
\end{comment}

\subsection{Dirac}
\begin{comment}
Discuss Dirac mechanism for mass
\end{comment}

\subsection{Majorana}
\begin{comment}
Discuss Majorana mechanism for mass
\end{comment}

\section{\zvbb searches}
\begin{comment}
Discuss \zvbb process and sensitivity to nature of neutrino.
Discuss concurrent sensitivity to hadron part
I feel like I should discuss ongoing searches but not in much detail?  Relevant information is: expected lifetime, mass, expected counts/year, expected limits?
Okay, yes.  Here is how this section could go: discuss the process and the resulting equation for the lifetime, and then talk about each of the components of the equation.  START with the discussion of the lifetime - can include details of ongoing experiments there.
\end{comment}
\begin{table*}
\centering
\ra{1.3}
\begin{tabular}{@{}llllll@{}}\toprule
experiment & isotope & mass [kg] & method & start/end & ref. \\
\midrule
past experiments \\
Heidelberg-Moscow & \Ge{76} & 11 & ionization & -2003 & \cite{} \\
Cuorcino & $^{130}$Te & 11 & bolometer & -2008 & \cite{} \\
NEMO-3 & $^{100}$Mo, $^{82}$Se & 7,1 & track + calorim. & -2011 & \cite{} \\
\vspace{0.1cm}

current experiments \\
EXO-200 & $^{136}$Xe & 175 & liquid TPC & 2011- & \cite{} \\
Kamland-Zen & $^{136}$Xe & 330 & liquid scint. & 2011- & \cite{} \\
GERDA-I/GERDA-II & \Ge{76} & 15/35 & ionization & 2011-/2013- & \cite{} \\
CANDLES & $^{48}$Ca & 0.35 & scint. crystal & 2011- & \cite{} \\
\vspace{0.1cm}

funded experiments \\
NEXT & $^{136}$Xe & 100 & gas TPC & 2015 & \cite{} \\
Cuore0/Cuore & $^{130}$Te & 10/200 & bolometer & 2012-/2015- & \cite{} \\
Majorana Demo & \Ge{76} & 30 & ionization & 2013 & \cite{} \\
SuperNEMO Demo/Total & \Se{82} & 7/100 & track + calorim. & 2014-/?? & \cite{} \\
SNO+ & $^{150}$Nd & 44 & liquid scint. & 2013 & \cite{} \\
\bottomrule
\end{tabular}
\caption{\zvbb experiments.  From \cite{}.}
\label{tab:experiments}
\end{table*}

\subsection{The Phase Factor}
\begin{comment}
Explain the phase factor, calculate it
\end{comment}


\subsection{Nuclear Matrix Elements}
\begin{comment}
Explain the NME - discuss different methods of calculation - QRPA, shell model, IBM, pairing
discuss uncertainties
discuss nuclear physics information that could help pin them down?
\end{comment}


\subsection{Mass Term}
\begin{comment}
Explain the mass term
\end{comment}


\subsection{The lifetime}
\begin{comment}
Discuss some different experiments working on \zvbb and the limits that are currently se on lifetimes and the expected limit for next-generation experiments.
\end{comment}

% % uncomment the following lines,
% if using chapter-wise bibliography
%
% \bibliographystyle{ndnatbib}
% \bibliography{example}
