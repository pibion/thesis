%
% Modified by Sameer Vijay
% Last Change: Tue Jul 26 2005 13:00 CEST
%
%%%%%%%%%%%%%%%%%%%%%%%%%%%%%%%%%%%%%%%%%%%%%%%%%%%%%%%%%%%%%%%%%%%%%%%%
%
% Sample Notre Dame Thesis/Dissertation
% Using Donald Peterson's ndthesis classfile
%
% Written by Jeff Squyres and Don Peterson
%
% Provided by the Information Technology Committee of
%   the Graduate Student Union
%   http://www.gsu.nd.edu/
%
% Nothing in this document is serious except the format.  :-)
%
% If you have any suggestions, comments, questions, please send e-mail
% to: ndthesis@gsu.nd.edu
%
%%%%%%%%%%%%%%%%%%%%%%%%%%%%%%%%%%%%%%%%%%%%%%%%%%%%%%%%%%%%%%%%%%%%%%%%


%
% Chapter 1
%

\chapter{NEUTRINO PHYSICS AND ITS DEPENDENCE ON NUCLEAR PHYSICS}
\label{chap:0vbb}
\section{A Brief Overview of Neutrino Physics}
\begin{comment}
Neutrino physics is fascinating because ??????.
Maybe I will talk about particles in general - but no, that's probably a bit too much.
I will mention the standard model table and talk about why neutrinos are important in physics today?

The neutrino was first proposed as a undetectable particle that carried away energy in nuclear decay processes \cite{Pauli}.  Twenty-six years later (fact check!), Reines and Cowan detected inverse beta decay to detect antineutrinos streaming out of a nearby nuclear reactor \cite{poltergeist}.  Detecting neutrinos is difficult: with an interaction volume of 10$^{26}$ potential targets, Reines and Cowan saw 1 event per week (FACT CHECK!!) \cite{poltergeist}.  Simply confirming the existence of the particle hypothesized to participate in beta decay was significant enough to merit the Nobel Prize \cite{CowanNobel}.

Figure: beta decay spectrum, with and without neutrino.  See F.A. Scott, Phys Rev 48, 391 (1935)

While detecting neutrinos is a difficult endeavor, it's also a lucrative one: particles that interact extremely rarely carry information about where they're created, no matter what they have to travel through to get to us.  With a detector that counts neutrinos, one can begin to imagine interrogating the cosmos: ``How many neutrinos are you making?  And you?  And you?''  Bahcahll was interested in finding out how many neutrinos the sun made.  And so he and Ray Davis set out to make a good neutrino counter. 

(AMY!  You're neglecting to talk about the experiment that demonstrated the left-handedness of neutrinos by Goldhaber in 1958.  This is kind of relevant to \zvbb!)
\end{comment}

The neutrino, a massless, chargeless particle, was first proposed by Pauli \cite{Pauli} as a means to preserve momentum conservation in nuclear beta decay.  The hypothesized process was
\begin{equation}
n \arrow p + e^- + \overline{v_e},
\end{equation}
where a neutron $n$ decays into a proton $p$, an electron $e^-$, and also an electron anti-neutrino $\overline{v_e}$, allowing the continuous electron energy spectrum that was observed. The existence of this difficult-to-detect particle was not confirmed until 26 years later, when Reines and Cowan used a nuclear reactor as a source of antineutrinos and observed inverse beta decay of proton targets \cite{poltergeist}.

\subsection{Neutrino Oscillation - the beginning}
In 1964 (CITE), the idea that the sun's energy source came from fusion between the nucleons in its plasma had not yet been tested.  John Bahcall, who had calculated the neutrino flux from the sun, teamed up with Ray Davis, who thought he may be able to count neutrinos using a large quantity of $^{37}$Cl - readily available in liquid form as dry cleaning fluid.  Neutrinos very rarely interact with matter, but it is possible for a neutrino and a $^{37}$Cl atom to inverse beta decay, producing $^{37}$Ar, which is radioactive.  Every few weeks, Ray Davis caciteully collected the Argon from the large tank of fluid and counted the radioactive $^{37}$Ar - about 7 atoms every two weeks \cite{Davis}.

But this was only about $\frac{1}{3}$ of the rate John Bahcall predicted should come from our burning star.  While Davis worked to characterize his detector and Bahcall worked to verify his calculations, physiscists were exploring the neutrino with different experiments that showed that neutrinos came in different flavors.

By 2001, physicists were confident that there were three flavors of light neutrinos (CITE Z-width experiment!) - the electron neutrino, $\nu_e$, the muon neutrino, $\nu_{\mu}$, and, last to be directly observed (CITE DONUT from FNAL!!), the tau neutrino, $\nu_{\tau}$.  If one assumed the neutrino had mass - a reasonable assumption (since presumably SOME EXPERIMENTS?? had put LOW LIMITS?? on its mass?  why else did we think this?), helicity is then always a good quantum number, with the neutrino (whose lepton number is +1) being left-handed and the anti-neutrino (lepton number = -1) being right-handed.  (AMY.  Left-handed-ness?  Right?  Helicity?  You need to define these terms and make them relevant!  Aaaaugh!!)

As Davis and Bahcall slowly convinced their peers that the discrepancy between their results represented real science, experiments began in earnest to find the sun's missing neutrinos.  (Gallex 1991, SAGE 1989, Kamiokande, SNO 1999)  GALLEX and SAGE were experiments that used $^{71}$Ga to detect solar neutrinos; inverse beta-decay on this nucleus has a lower threshold than does $^{37}$Cl, but even with sensitivity to lower-energy neutrinos, the problem of missing neutrinos persisted.  What was needed was a detector that was sensitive to all flavors of neutrinos rather than only electron neutrinos, and in 1999 the Sudbury Neutrino Observatory (SNO) began taking data.

Experiments that rely on inverse beta decay caused by solar neutrinos are only sensitive to electron neutrinos because even the most energy neutrinos can be created with in the sun is 11 MeV (from hep? and 8B? CITE, yo).  The mass of $^{37}$Cl is 34433.5165 MeV while the mass of $^{37}$Ar is 34434.3034 MeV; the mass difference is 0.8139 MeV.  But to satisfy energy conservation, the neutrino must have not only enough energy to make up this mass difference, but also enough energy to produce an electron, bringing the total required energy to a minimum of 1.3249 MeV for this process.  While the $^{71}$Ga used in GALLEX and SAGE required much less energy of its neutrinos for inverse beta decay (only 0.2326 MeV for the mass difference between nuclei!), it is easy to see that the even the most energetic of solar neutrinos  

\subsection{Neutrino Oscillation - current known mixing parameters}
\begin{comment}
Talk about current constraints on parameters
Talk about experiments that constrain these parameters?  This seems far afield, maybe.
\end{comment}


\section{Majorana vs Dirac}
\label{sec:mVd}
\begin{comment}
Discuss mechanisms by which neutrinos could get their mass.  
\end{comment}


\subsection{Majorana}
\begin{comment}
Discuss Majorana mechanism for mass
\end{comment}


\subsection{Dirac}
\begin{comment}
Discuss Dirac mechanism for mass
\end{comment}


\section{\zvbb searches}
\begin{comment}
Discuss \zvbb process and sensitivity to nature of neutrino.
Discuss concurrent sensitivity to hadron part
I feel like I should discuss ongoing searches but not in much detail?  Relevant information is: expected lifetime, mass, expected counts/year, expected limits?
Okay, yes.  Here is how this section could go: discuss the process and the resulting equation for the lifetime, and then talk about each of the components of the equation.  START with the discussion of the lifetime - can include details of ongoing experiments there.
\end{comment}


\subsection{The Phase Factor}
\begin{comment}
Explain the phase factor, calculate it
\end{comment}


\subsection{Nuclear Matrix Elements}
\begin{comment}
Explain the NME - discuss different methods of calculation - QRPA, shell model, IBM, pairing
discuss uncertainties
discuss nuclear physics information that could help pin them down?
\end{comment}


\subsection{Mass Term}
\begin{comment}
Explain the mass term
\end{comment}


\subsection{The lifetime}
\begin{comment}
Discuss some different experiments working on \zvbb and the limits that are currently se on lifetimes and the expected limit for next-generation experiments.
\end{comment}

% % uncomment the following lines,
% if using chapter-wise bibliography
%
% \bibliographystyle{ndnatbib}
% \bibliography{example}
