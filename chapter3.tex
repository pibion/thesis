%
% Modified by Sameer Vijay
% Last Change: Wed Jul 27 2005 13:00 CEST
%
%%%%%%%%%%%%%%%%%%%%%%%%%%%%%%%%%%%%%%%%%%%%%%%%%%%%%%%%%%%%%%%%%%%%%%%%
%
% Sample Notre Dame Thesis/Dissertation
% Using Donald Peterson's ndthesis classfile
%
% Written by Jeff Squyres and Don Peterson
%
% Provided by the Information Technology Committee of
%   the Graduate Student Union
%   http://www.gsu.nd.edu/
%
% Nothing in this document is serious except the format.  :-)
%
% If you have any suggestions, comments, questions, please send e-mail
% to: ndthesis@gsu.nd.edu
%
%%%%%%%%%%%%%%%%%%%%%%%%%%%%%%%%%%%%%%%%%%%%%%%%%%%%%%%%%%%%%%%%%%%%%%%%

%
% Chapter 3
%

\chapter{TWO PROTON TRANSFER AT NOTRE DAME}
\label{chap:2pExpt}

Give an overview of the requirements for two-proton transfer and say that ND has a buncher and a Tandem accelerator that goes up to 10 MV so we can get beam energies up to 20 MeV for 3He and we have a beamline with a long flight path SO we can do this experiment.

Will discuss the elements that contribute to crucial aspects of the experiment

When studying two-proton transfer with 3He beam, the neutron flying off is the easiest object to study as the heavy nucleus (now with two additional protons!) has little hope of making its way out of the target.  While charged particles are often simple to detect, neutrons are not, and the way we detect them forces design constraints on our experiment.

Imagine for a moment the options that are available to the neutron that wishes to interact with any detector.  It has no charge, so disturbing electrons via the electromagnetic force is not possible.  The neutron can interact with the electrons weakly, but this probability is impractically small.  A neutron may also interact with a proton, strongly.  This mechanism happens frequently enough to be of use to us.  A neutron that transfers some of its energy to a proton can rely on the proton, in all its charged glory, to register a signal in the detector.  What experimenters must do is provide plenty of protons for neutrons to collide with - large quantities of water or plastics, with their long hydrocarbon chains, are popular choices.

What does a neutron signal look like in such a detector?  If the neutron always deposited all its energy in the detector, the energy spectrum would show sharp peaks corresponding to each neutron energy.  Sadly, a neutron interacting with a proton rarely transfers all its energy to that proton.  In this transfer experiment, it is necessary to identify signals coming from neutrons - worse, specific neutrons!  The neutrons of interest are those coming from the ground state of the daughter nucleus, not neutrons from its many excited states.  Because the neutron's deposited energy does not determine that neutron's full energy, the experiment must be sensitive to some quantity that gives unambiguous information about the neutron's energy.

% figure of charged-particle energy spectrum
% figure of neutron energy spectrum

Time is uniquely determined by the neutron's energy - the time it takes for the neutron to travel from the target to the detector depends on its energy as in \eq \ref{eq:TOF}.  

\begin{equation}
\frac{v}{c} = \sqrt{\frac{E^2 - m^2c^4}{E^2}}
\label{eq:TOF}
\end{equation}

Measuring the time of flight (TOF) of the neutron with our detector requires four things.  In a continuous beam, there is no way to know which 3He was associated with a neutron event in the detector, and therefore no way to determine TOF.  Bunching the beam so that clumps of 3He arrive at the same time and providing a signal correlated to their arrival at the target allows a TOF measurement with a precision determined by the time-width of the bunch.  This limit on time resolution constrains the distance between the target and detector.  In this experiment, it is necessary to distinguish between neutrons whose energies differ by only 0.5 MeV.  With a flight path of 1 m between the target and detector, the difference in time between these two neutrons is ?? ns - far less than the resolution determined by the beam bunch.  The flight path must be long enough to distinguish ground state neutrons from neutrons associated with the first excited state of the residual nucleus.

\section{Beam Bunching}
Discuss the operation of the buncher, pulse selector, and sweeper.  Discuss beam loss, which actually isn't too bad since we're radiation-limited in the target room.
Time resolution ~ 1ns which again isn't too bad since our detector has similar resolution.
Discuss modifications made to buncher platform?

The beam buncher at Notre Dame works by slowing down particles that would arrive too early at the target and speeding up particles that would be arriving too late.  To acheive this, two grids perpendicular to the beam connect to a power supply create an electric field that varies in time.  Bunching all the beam requires that the electric field be a triangle wave in time, but commercially available power supplies providing adequate current and rapid enough signal generally vary sinusoidally in time.  At some beam facilities \cite{LynchBunching}, multiple power supplies provide additional frequncies, better approximating a triangle wave.  At Notre Dame, one frequency only - 11 MHz is used.  The best bunching occurs when the wave is at its steepest; when the electric field is changing very slowly, little bunching occurs.  The effect of the bunching, then, is that the beam is continuous, with large bunches every 101 ns.  The beam in between these bunches would render our time signal useless, and must somehow be removed.

The ``sweeper'' provides a large electric field that ramps up and down very quickly to remove the unwanted beam between the bunches.  While conceptually simple, the time scale required for the charging of the plates makes it difficult to find a commercially available power supply.  The field must turn "on" on a timescale much smaller than the beam bunch, or else become the limiting factor in the time-width of the bunch.  The power supply used provides ?? charge in ?? ns.

%figure of beam profile and sweeper

The beam buncher and sweeper are all that is needed to provide bunched beam.  At Notre Dame, the beam bunches are 101.52 ns apart and are typically 1 ns to 2 ns wide.  This bunch spacing presents a problem for the experiment because the neutron TOF is in excess of 200 ns.  With bunches arriving at the target every 100 ns, it will be impossible to tell if a neutron is very slow and associated with bunch A or very fast and associated with bunch B.  The spectrum will be considerably complicated.

% figure: the considerably complicated spectrum

To give all the neutrons resulting from one beam bunch time to reach the detector before the next bunch strikes, a ``pulse selector'' knocks out three of every four bunches, resulting in 300 ns between each bunch.  Even with such generous time between bunches, there are very slow neutrons from previous bunches that overlap with gammas from the current bunch, but these vanish with even a low energy cut.

% figure: slow neutrons and their dissappearance

The downside to the bunching system is beam loss.  However, the radiation limits on the room with the detector are so low that even with pulse selection, the beam is at its maximum allowed current.

\section{Acclerator}
The accelerator at Notre Dame is a Van deGraff accelerator made by High Voltage Engineering.  Its maximum potential is 10 MV and is a tandem Van deGraff - a thin Carbon target in the center of the machine strips electrons, providing additional accelartion to the beam.  The ion source must provide negatively charged ions to the Tandem's first acceleration stage.

A dipole magnet at the end of the Tandem selects beam of the desired energy.  This magnet is tuned using the NMR frequency of Hydrogen (?) and has an error of less than ?? (units?).  Comparing this energy spread to the energy spread introduced by the target gives a sense of how small the error associated with the beam energy is; the target spreads the energy by 0.5 MeV.


\section{Beam Focusing}
Discuss beamline: steering magnets and focusing quadrupoles.
Explain focusing of solenoids.
Beam spot size much smaller than target - what are the dimensions?

After energy selection, the beam travels through two target rooms before reaching our detector.  A variety of focusing and steering elements keep the beam colimated and in position.  Quadrupole triplet and doublet magnets are the most common beam focusers.  There are two Einzel Lenses that provide electrostatic focusing early in the beam line as well.  Steering plates before and after the accelerator help correct the beam trajectory, although additional steerers in the East Target Room are needed as well.  The large solenoid magnets are the final focusing element before the target and focus the beam to a spot approximately 2 mm in diameter.

% figure: beamline

\section{The Detector}
Discuss neutron wall briefly.  Can reference NIMA paper. Explain why it's important that it's wide-angle. 
Electronics diagram!  Discuss two most important aspects: TDC and ADC from phototubes

The neutron detectors are large strips of high density polyethylene (HDPE) plastic doped with scintillator (what kind of scintillator??).  Sixteen independent detectors sit in a rough circle around the target, with the forwardmost angle being 5$\circ$ and the largest angle being $21\circ$.  The angle step size between each detector is approximately 0.7$\circ$.  The detectors sit 15.6 m away from the target to ensure a resolution of at least 0.5 MeV at 22 MeV.  This distance comes at a steep price; the solid angle is only (approx. 10 cm/15 m x 2.5 m/15 m = 0.11) sr, or 0.88\% of the unit circle.

% figure - picture of detector

The plastic scintillator, equipped with fast timing PMT's, are sensitive to energy deposition and time.  Energy deposited is not useful for neutron identification; time is the information that allows us to ID them.  In principle, the timing information is all the DAQ needs to record.  This would be true if only there were no background radiation!  The low-energy gamma radiation from the concrete leaves signal in the detector at a high rate - much higher than the rate of beam! Measuring the Energy deposition is crucial because it allows us to throw out low-energy background from the room.  

What is clear is that while energy information is necessary, the energy information does not need to be terribly precise.  The timing information, however, must be as precise as possible.  The goal with the electronics is to not add spread to the timing that is noticable above the timing spread already inherent in the beam bunching.  The detectors are equipped with photomultiplier tubes (PMT's) at the top and bottom that have excellent timing response - their risetime is approximately 5 ns.  Constant Fraction Discriminators (CFD's) give timing information with jitter that's less than 1 ns.


Flight path and time resolution as a function of neutron energy - apply this to Ge ground and first excited state!
Beam monitor
Dead time
Include a sample calculation?  Like: this is what we see in the detector.  To get an absolute cross-section, here is the calculation:
counts * time * particle current * target thickness
So we get the counts this way and the time with the signal from the DAQ and the target thickness from Hope

\subsection{Testing with 26Mg}
Show results from first run and look at timing - hey it's all right!
Look at background - will need to improve

% % uncomment the following lines,
% if using chapter-wise bibliography
%
% \bibliographystyle{ndnatbib}
% \bibliography{example}
