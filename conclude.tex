%
% Modified by Sameer Vijay
% Last Change: Wed Jul 27 2005 13:00 CEST
%
%%%%%%%%%%%%%%%%%%%%%%%%%%%%%%%%%%%%%%%%%%%%%%%%%%%%%%%%%%%%%%%%%%%%%%%%
%
% Sample Notre Dame Thesis/Dissertation
% Using Donald Peterson's ndthesis classfile
%
% Written by Jeff Squyres and Don Peterson
%
% Provided by the Information Technology Committee of
%   the Graduate Student Union
%   http://www.gsu.nd.edu/
%
% Nothing in this document is serious except the format.  :-)
%
% If you have any suggestions, comments, questions, please send e-mail
% to: ndthesis@gsu.nd.edu
%
%%%%%%%%%%%%%%%%%%%%%%%%%%%%%%%%%%%%%%%%%%%%%%%%%%%%%%%%%%%%%%%%%%%%%%%%

%
% Chapter 4
%

\chapter{CONCLUSION}
\label{chap:conclude}
\begin{comment}
Review the importance of accurate NME for \zvbb searches
Review the importance of two-nucleon transfer reactions in general and two-proton transfer reactions to NME calculations

Thanks to the construction of a cosmic veto shield, we were able to obtain a measurement of the 74,76Ge(3He,n) cross section (300 $\mu$ barns/sr at 0$\circ$, 150 $\mu$ barns/sr at 0$\circ$, respectively) and place a limit on excited 0+ states to within 15\%.

The implication of this work for NME calculations are that ????  Dude seriously you need to be able to say something about this.
\end{comment}
An observation of \zvbb would confirm that neutrinos are Majorana fermions and also allow the calculation of the neutrino mass scale.  However, determining the mass scale from the \zvbb lifetime requires knowing \NME, calculations of which can vary by as much as a factor of 5.  Single-nucleon transfer experiments can give information on valence shell occupancies and vacancies and have already helped reduce the spread in the \NME for \Ge{76}.  However, the \zvbb process would occur primarily on highly-correlated neutron pairs, and single-nucleon transfer is not sensitive to pairing in the nucleus.  Two-nucleon transfer experiments can give information on ground-state nucleon pairing, which is particularly important to QRPA, one of the leading methods in \NME calculations.

Two-nucleon transfer experiments have been completed or nearly so for several candidate nuclei.  The $^{130}$Te candidate is an interesting case because both two-proton transfer on $^{128}$Te and two-neutron transfer onto $^{130}$Te have been studied \cite{protonPairsTellurium,neutronPairsTellurium}.  While no excited \zp states were populated in the neutron-pair transfer, the proton-pair transfer populated an excited \zp state with 30\% the strength of the ground state.  This suggests that the proton-pairing strength is split between the ground state and at least one excited \zp state.   The work that has been done on \Ge{76} has shown that the neutron-pairing strength is concentrated in the ground state \cite{neutronPairsGermanium}, but this offers no constraint on the proton-pairing in \Se{76}.  Investigating the proton-pairing in \Se{76} and, as a check, \Se{78} is the work of this thesis.  The reaction \reaction was used to look for excited \zp strength.  No excited \zp states were observed in either nucleus, and limits on such states were determined to be 4-8\% of the ground-state cross section for \GeReaction{74}{76} and 8-20\% of the ground-state cross section for \GeReaction{76}{78}, depending on excitation energy.  

The zero-degree cross sections for both \GeTargets were determined using a DWBA fit.  It was found that the ground-state cross section for \Ge{74} is $360\pm??$~$\mu$b/sr and for \Ge{76} is ??, where these errors do not include the 10\% systematic error due to uncertainty in the efficiency.  While no excited states for \Ge{76} were observed, the difference in cross section prompted an investigation to determine if the decline was an expected result of kinematics or if excited \zp states had been missed in the analysis.  All DWBA calculations confirmed that the trend was consistent with the expectations of the reaction model and not due to missing ground-state \zp strength. 


\section{Future Work}
\begin{comment}
Weaknesses of this work are: 
1. STATISTICS LIMITED AARGH!!!  liquid scintillator would be helpful in reclaiming low-energy transfer neutron hits which could improve statistics by ~30\% (<- this is NOT currently precise!)
2. momentum mismatching between the pair transfer we're looking at and pair correlations most relevant to \zvbb?  
3. ?? come on there must be other issues

Future work that would be helpful would be
1. looking at this pair transfer on other targets
2. leptonic probes that could excite pair correlations at higher momenta? \cite{LeptonPP}
\end{comment}
That no evidence has been found for proton-pair strength in excited \zp states of \GeTargets is encouraging.  However, \zvbb searches use not just \Ge{76} as a target, but also $^{48}$Ca, \Se{82}, $^{100}$Mo, $^{130}$Te, $^{136}$Xe, and $^{150}$Nd.  The Mo and Te isotopes have been studied with both proton-pair and neutron-pair transfer, but proton-pair transfer data is still needed for the candidate Ca, Se, Mo, and Nd isotopes.  While the detector used in this experiment may be able to study $^{46}$Ca(\He{3},n), isotopes with masses higher than \GeTargets would be difficult to investigate without increased timing resolution, which is not possible at present.  This study does suggest that DWBA predictions match the measured ground-state cross sections of the $f-p-g$ nuclei.  If these cross sections were measured, it may be possible to place a limit on expected excited \zp strength based on the deviation from the predicted ground-state cross section.

% % uncomment the following lines,
% if using chapter-wise bibliography
%
% \bibliographystyle{ndnatbib}
% \bibliography{example}
